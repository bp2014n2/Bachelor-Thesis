\section*{Preface}
\addcontentsline{toc}{section}{Preface}

The relevance of computer science is increasing for every section of our lives. Today, various information about our medical condition are collected whenever we interact with the health sector. When visiting a doctor or getting treated in a hospital, medical health records store these information. This data remains rather unused, but analyzing it can unveil information that were previously hidden. The generated knowledge can be used to optimize health care by detecting diseases earlier or finding adverse reactions of medications.
Developing a system that allows clinical researchers intuitively analyze big medical data with the described capabilities is a complex task. Analyzing millions of clinical findings is a big challenge as well on the interface side where the user should not need to know anything about underlying technologies, but also for the server component which manages huge data sets. Traditional database systems do not provide sufficient access to such large datasets. Therefore, new and better scaling database systems are required to fit our requirements.
\i2b2\footnote{\url{https://www.i2b2.org/}} is a software developed at Harvard Medical School which allows physicians to easily analyze medical data without having any technical background. To analyze this data in detail and match the previously described goals, we created an interface for using R embedded in \i2b2.\\

The work on this target is spread over three bachelor theses. How to efficiently visualize data in order to detect hidden information is described by Carl Ambroselli in \textit{“Visualizing Information of Big Medical Data}. Philipp Schirmer describes in \textit{Risk Prediction for Big Medical Data using Logistic Regression} how to learn from past data and apply the generated knowledge to make predictions about future diseases and illnesses. Built up on this technique,  Jan-Peter Heuzeroth  evaluates medications by comparing different groups of patients in \textit{Evaluation of Treatment Effectiveness through Propensity Score Matching on Big Medical Data}.
To have very fast access to the analyzed data, we integrated a distributed database system called HPCC into \i2b2. The work on this project part is split into three theses. Marc-Philipp Bismar explains how querying the data for \i2b2 can be speeded up using either a relational or a distributed database system in \textit{Database Optimization for Big Medical Data and i2b2}. How to adapt \i2b2’s standard SQL-interface for HPCC’s own ECL query language, Sven Lehmann expounds in \textit{A federated database system with support for HPCC by using SQL}. In \textit{Optimization and Extension of the HPCC JDBC Driver} Peter Schwarz describes how to design a JDBC driver efficiently for big data analytics.